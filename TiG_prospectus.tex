\documentclass[11pt]{article}
\usepackage{color}
\usepackage{authblk}%allows footnote format for authors
\usepackage[letterpaper, margin=1in]{geometry} %package that allows changes in margins and header/footers
\usepackage[numbers,sort]{natbib}
\bibliographystyle{ieeetr}
\newcommand{\mbh}[1]{\textcolor{blue}{ \emph{\scriptsize  #1}} } %creating command for Matt's comments
\newcommand{\lwang}[1]{\textcolor{red}{ \emph{\scriptsize  #1}} } %creating command for Li's comments
\newcommand{\gmj}[1]{\textcolor{green}{ \emph{\scriptsize  #1}} } %creating command for Garrett's comments

\title{Review: Crop adaptation through historical introgression from wild relatives}

\author[1]{Authors: Garrett M. Janzen}%author information
\author[1]{Li Wang}
\author[1,*]{Matthew B. Hufford}
\affil[1]{Department of Ecology, Evolution, and Organismal Biology, Iowa State University, Ames, Iowa, USA}
\affil[*]{Correspondence: mhufford@iastate.edu (M.B. Hufford)}
\date{}

\begin{document}



\maketitle

The traditional paradigm in crop domestication has been origin from a wild relative within one or more defined geographic centers followed by expansion to the modern-day extent of cultivation.
Absent from this paradigm are the effects of hybridization between diffusing domesticates and closely-related, locally-adapted wild relatives outside the center of origin.
New methods have recently been employed to detect genome-wide patterns of introgression in a number of species.
In this review, we will: 1) briefly describe these methods and provide a summary of their recent application for detection of crop-wild introgression, 2) review evidence supporting the hypothesis that wild-to-crop introgression has conferred local adaptation, 3) consider how the prevalence of this introgression alters traditional concepts of domestication, and 4) describe future advances in both basic and applied genetics that can be made through the study of introgression in agroecosystems.
\lwang{general comments: A. Could this be too long? I have no idea about the requirement (300-600 words). Is that okay if we listed the outline of each section? B. Some references are kind of old. They require references in the past 2-4 years. }

\section*{Introgression methods and their recent applications}
\lwang{The examples listed here are not crop-wild introgression, which is contradictory with the first point in the abstract part. "briefly describe these methods and provide a summary of their recent application for detection of crop-wild introgression". When I red here, I felt my attention was diverted. Maybe just one or two sentences.}
\gmj{I could not find the word count limit on the TiG website, but if we need to cut down, Item 2. of this section might be a place to do so. a-d could perhaps be reduced to a single sentence, for example.}

The recent availability of genome-wide resequencing and reduced-representation genotyping (\emph{e.g.}, GBS and RAD-Seq) data combined with new analytical methods have facilitated comprehensive study of introgression across a number of species (\textbf{Table 1}).
\begin{enumerate}
\item{High-density marker data can be used with haplotype-based and other methods to assign specific genomic regions to a taxon of origin and to identify introgression across taxa \cite{Martin2015,Price2009,Lawson2012,pease2015,rosenzweig2016,geneva2015}.}
\item{In several instances, introgression has been found to be conserved across individuals and populations suggesting an adaptive role:}
	\begin{enumerate}
	\item{Adaptive introgression has been discovered from Neanderthals and other archaic hominins into humans at loci controlling skin pigmentation, defense against pathogens, and tolerance of high altitude (reviewed in \cite{Racimo2015}}).
	\item{Introgression across butterfly species at protective color-pattern loci has conferred M\"{u}llerian mimicry (\cite{Heliconius2012}; \textbf{Figure 1})}.
	\item{Adaptive introgression across mosquito species has spread insecticide resistance \cite{Norris2015}}.
	\item{Introgression across \emph{Mimulus} (\emph{i.e.}, monkeyflower) species has resulted in adaptation to pollinator preference and has contributed to speciation \cite{Stankowski2015}}.
	\end{enumerate}
\item{Identification of introgression across a wide range of taxa at loci controlling a variety of adaptive traits suggests it has been an important evolutionary force.}
\end{enumerate}

\noindent{\textbf{Table 1:} List and brief description of recently developed methods and examples of empirical studies employing these methods.}

\noindent{\textbf{Figure 1}: Wing coloration patterns in \emph{Heliconius} and evidence for introgression across species based on Patterson's \emph{D}-statistic; adapted from \cite{Heliconius2012}.}

\lwang{In this part, maybe better to incorporate this reference, Detection and Polarization of Introgression in a Five-Taxon Phylogeny; and a new method published this year RND\_min Powerful methods for detecting introgressed regions from population genomic data; I donot know if we should also include G\_min, that is the statistic we tried that didnot work well for us: A new method to scan genomes for introgression in a secondary contact model. I put the three in the bib file.}

\section*{Crop adaptation through introgression}
Over the last few years, several high-profile publications based on genome-wide data have documented introgression between crops and wild relatives outside putative domestication centers. A history of introgression during diffusion appears to be the rule for crops rather than the exception.
\begin{enumerate}
	\item{Crops, given their frequent history of diffusion from defined centers of origin, are ideal recipients of adaptive introgression. Theory suggests that such colonizing species will overwhelmingly be recipients rather than donors of introgression \cite{Currat2008}.}
	\item{Empirical studies have revealed that introgression has occurred in many of the world's most important crops (\textbf{Table 2}).}
	\begin{enumerate}
		\item{Maize: As maize spread from a lowland center of origin into the Mexican highlands, it received introgression from a highland-adapted wild relative \cite{Hufford2013}.}
		\item{Sunflower: Domesticated sunflower has received substantial introgression from wild relatives post-domestication that has potentially reintroduced a branching morphology \cite{Baute2015}.}
		\item{Rice: Introgression appears to have played an important role in the history of rice.  Japonica rice cultivars were likely domesticated first from wild rice populations in southern China, and indica cultivars later developed through hybridization of ancient japonica with new wild rice populations in south and southeast Asia \cite{Huang2012}.}
		\item{Barley: Landraces of barley show shared ancestry with geographically-proximate populations of wild relatives, suggesting introgression \cite{Poets2015}.}
		\item{Olive: Hybridization between wild and domesticated olive has been prevalent throughout the Mediterranean Basin \cite{Diez2015}.}
		\item{Cassava: Substantial introgression from a closely-related wild species has been found in cassava, occurring both naturally and through a targeted breeding program for disease resistance \cite{Bredeson2016}.}
	\end{enumerate}
	\item{Predictions regarding the likelihood of adaptive introgression in additional crops can be made through comparison of their centers of origin to both their current extent of cultivation and the distribution of wild relatives across environmental gradients (\textbf{Figure 2}).}
\end{enumerate}
\lwang{I am afraid that most of the crop examples actually did show evidence of introgression, but few supplied strong evidence of adaptive introgression. I have not red all of these paper, maybe it is just my prejudice. }
\gmj{Not all crops that I have researched are included in this list.  Are we only here mentioning the strongest cases, or will the weaker cases be excluded from the paper altogether?  Also, regarding Li's comments above, many times there is indeed evidence that the introgression is adaptive.  This might be most apparent when the introgression confers pest resistance alleles, but also in other cases as well.  That being said, perhaps the caveat mentioning the distinction between adaptive and neutral introgression should be made more explicit?}

\noindent{\textbf{Table 2:} Overview of literature evaluating potential for crop-wild introgression in the world's 15 most important crops including a column of references and a column summarizing major findings.}

\noindent{\textbf{Figure 2}: Multi-panel figure showing maps of crop centers of domestication, distributions of wild relatives, and current extent of cultivation across gradients of (a) temperature, (b) precipitation, and (c) elevation. This figure will illustrate the adaptive potential of wild-to-crop introgression.}
\gmj{Will we have to generate these images before we submit the prospectus?}

\section*{Reevaluating concepts of domestication}
A framework in which crops were domesticated from a single population or even a single species is, in several instances, an oversimplification. A demography incorporating introgression from additional sources appears to be more correct for many crops. With this in mind, certain aspects of crop evolution must be reevaluated:
\begin{enumerate}
	\item{The initial domestication bottleneck is most likely underestimated when introgression is not considered. Chromosomal regions experiencing introgression will potentially have increased effective population size relative to non-introgressed regions. Inferences regarding the strength of domestication bottlenecks should be revisited with this in mind.}\lwang{if the donor population has a relatively small population size, then the domestication bottleneck could be over-estimated.}\gmj{is the Kelley Harris, Rasmus Nielsen paper on Neanderthal genetic load in humans (the old CAMP paper) an example of this, Li?  I just want to make sure I understand.}
	\item{The time since domestication based on levels of sequence divergence may be inflated when introgressed haplotypes from divergent taxa are included in estimates.}\lwang{It could be either inflated or underestimate,. It depends on the introgressed haplotype are more or less divergent from the wild progenitor, right?}
	\item{The influx of diversity through introgression may belie selection occurring on these genomic regions during domestication or local adaptation since one of the hallmarks of selection is reduced diversity.} \lwang{I donot quite understand this point. So, it is assumed that the introgressed haplotype has higher diversity, which renders difficulty to detect selection in such regions, right? But the case in maize is that even the introgressed haplotypes had higher diversity, but as selection has act long time on the region, the observed current diversity has already been reduced.}
	\gmj{It may not fit here perfectly, but you might consider a point about how crop-wild introgressions can also mask or obscure even the true progenitor species and center(s) of domestication.  The examples of potato and tomato domestication comes to mind; in potato, persistent and thorough interbreeding with a complex of wild relatives during domestication probably make it impossible to identify a single progenitor (if ever there was one), and in tomato, widescale recent introgression between two or three wild relatives confound attempts to identify which is the progenitor, and therefore the center of domestication as well.}
\end{enumerate}

\section*{Future studies in crop-wild introgression}
Additional study of introgression in agroecosystems could lead to advances in both basic and applied genetics:
\begin{enumerate}
	\item{Basic questions:}
		\begin{enumerate}
		 	\item{To what extent does the level of introgression across taxa depend on divergence time between donor and recipient taxa?} \lwang{maybe put alternative hypothesis here, or due to relative mutation load of the donor and recipient species?}
		 	\item{At what geographic scale does adaptive introgression occur? Is introgression frequently restricted to very local populations or is it often seen over broad geographic ranges?}\gmj{Maybe not just geographical ranges, but also across environmental clines (e.g. temperature, precipitation, altitude, day lengh, etc.)?}
		 	\item{To what extent have domesticates served as bridges for gene flow between previously allopatric taxa?} \lwang{I feel the second and the third points could be combined. How the wild species could introgress into the allopatric landraces? Does domesticates serve as bridges?}
		\end{enumerate} 	
	\item{Applications:}
		\begin{enumerate}
		 	\item{Loci underlying the domesticated phenotype that are potentially beneficial targets for crop improvement can be more clearly identified by removing the confounding population genetic signal of introgression.} \lwang{I am not quite clear with this point. It will be interesting to see that the domestication loci have further receive gene flow from the other wild relatives and it has not been purged out, as I think the domestication loci undergo strong positive selection, thus also strong purifying selection to dispel influx gene flow. Right?}\gmj{Removing the signal of introgression, is this more likely to uncover new and interesting domestication loci, or will it more likely reveal that some of the previously-identified loci of interest are not very interesting after all (as in, the identification of the introgression changes our understanding of the history of the locus and there is less evidence of selective pressures on it), or equally either case?  I hope that question made sense.}
		 	\item{Adaptive introgression that is clearly tied to a specific environment may include beneficial alleles that can be utilized in crop breeding.}\gmj{This point is very clear, but should it be fleshed out further?  It wouldn't be difficult to include some examples.}
		 \end{enumerate}
\end{enumerate}


\bibliography{TiG_references}

\end{document}
