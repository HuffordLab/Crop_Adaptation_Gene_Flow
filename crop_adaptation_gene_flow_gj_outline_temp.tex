\documentclass[11pt]{article}
\usepackage{color}
\usepackage{authblk}%allows footnote format for authors
\usepackage[letterpaper, margin=1in]{geometry} %package that allows changes in margins and header/footers

\newcommand{\mbh}[1]{\textcolor{green}{ \emph{\scriptsize  #1}} } %creating command for Matt's comments
\newcommand{\lwang}[1]{\textcolor{red}{ \emph{\scriptsize  #1}} } %creating command for Li's comments

\title{Crop adaptation through gene flow (outline)}




\begin{document}

\maketitle

\section*{Outline}
\subsection*{Outline parts}
Introgression in Crops (General Information)
Crop Examples
    Maize
    Wheat/Rye
    Barley
    Sunflower
    Tomato?
Non-Crop System Examples?

1\subsection*{Introgression in Crops (General)}

From {label22}
A very good review.
Hybrid zones make for a great model system of study.
Introgression from wild populations into agronomic populations is potentially beneficial for breeding programmes.
It is generally easier to introgress into crops than introgressing into wild populations (stronger selection, overcoming LD).

From {label32}
Another very good review, with more of a molecular emphasis

From {label34}
The difficulties of utilizing wild relatives in breeding programmes effectively.

From {label6} {label7} {label26} {Label27}
Lots of papers focus on gene flow from crop to wild relative, particularly about transgene escape

\subsection*{Maize}

From {label8}
"All things considered, there appears to be little or no clear-cut evidence to support the idea that the teosintes have been greatly altered by maize introgression."

From {label7}
Lists papers citing the sources of barriers to gene flow between teosinte and maize

From {label1}, {label2}, {label20}
Nobogame Valley, Mexico, where teosinte genes flow into local maize.  This gene flow was measured morphologically and imparted positive agronomic results.

From {label10}
"Several races of maize that contain isozyme alleles that have apparently been transferred to maize from subsp. mexicana by introgression." (quote from {label9}

From {label4}
Table 4, STRUCTURE analysis of admixture in maize, source and destination included (gene flow between teosinte and maize, both directions)

From {label11}
Gene flow is probably greater from maize to teosinte.
Certain morphological and flower timing characterists promote genetic isolation between maize and teosinte.
Gene flow from maize to teosinte occurs most easily when teosinte pollinates maize

From {label12}
Gene flow from Parv and "Chalco" teosinte into maize for the past 200 years, may still be ongoing
These teosintes should be preserved for their potential to fuel new breeding programs.

From {label13} {label14}
Maize-maize introgression
Research on introgression between highland and lowland maize shows adaptive traits being transferred

From {label16}
Gene flow between mexicana and maize is low enough that bad alleles are unlikely to be passed, only neutral/beneficial alleles. (from Slatkin 1987)

From {label17}
mexicana introgression into highland maize landraces
introgression into maize favored over introgression into teosinte

From {label18}
higher mexican gene flow into landraces Cacahuacintle, Palomero de Jalisco, and Palomero Taloqueno, but also into mexican maize (non-landrace) at the same elevation

From {label19}
Similar to label18 and label17, mexicana and parviglumis into maize (simulation)

From {label32}
most common with mexicana (introgression is)

\subsection*{wheat}

From {label24}
review/overview.  See table 1

From {label25}
Wheat was given leaf rust resistance genes from wild populations (summarized in Table 1)

From {label26}
A few examples (wild genes into wheat) given.  Originally, only with cross-compatible species, now with cross-incompatible species.
Cites and tests several QTLs

From {label31}
"suggestions are made concerning techniques for exploitation of the wild diploid species in wheat breeding programs."

\subsection*{barley}

From {label24}
review/overview.  See table 1

From {label30}
"wild barley does harbour valuable alleles, which can enrich the genetic basis of cultivated barley and improve quantitative agronomic traits."

\subsection*{sunflower}

From {label33}
yeah, introgression with wild relatives may be common, unsure about direction
intermediates found in the field

\subsection*{non-crop systems}
From {label3}
Bacteria, malaria mosquitoes, blackflies, Darwin's finches, butterflies (each example has a cited source, not included in the bibliography of this outline).
Also, non-crop plants, like trees (oak, larch).

From {label15}
Helianthus, Iris
Mouse, Salamander

From {label21}
Iris, Helianthus, Cowania/Purshia
Dacus (now Bactrocera), Anopheles
Haemophilus influenza, Trypanosoma cruzi

From {label23}
165 proposed cases of introgression, 65 "deemed to be sufficiently documented" (plants)

From {label28}
Milkweeds

From {label29}
White clover












\begin{thebibliography}{1}
\bibitem{label1} H. G. Wilkes (1977) 
Hybridization of Maize and Teosinte, in Mexico and Guatemala and the Improvement of Maize. 
{\em Economic Botany\} 31:254-293

\bibitem{label2} C. Lumholtz (1902)
Unknown Mexico. Charles Scribner's Sons, New York. Vol I. 530 pp.
%This citation was problematic

\bibitem{label3} J. Mallet (2005)
Hybridization as an invasion of the genome.
{\em TRENDS in Ecology and Evolutin\} 20:229-237

\bibitem{label4} J. Ross-Ibarra, M. Tenaillon, B. S. Gaut (2009)
Historical Divergence and Gene Flow in the Genus Zea.
{\em Genetics\} 181:1399-1413

\bibitem{label5} N. C. Ellstrand, H. C. Prentice, J. F. Hancock (1990)
Gene Flow and Introgression from Domesticated Plants into their Wild Relatives.
{\em Annual Review of Ecology and Systematics\} 30:539-563

\bibitem{label6} D. Quist, I. H. Chapela (2001)
Transgenic DNA introgressed into traditional maize landraces in Oaxaca, Mexico.
{\em Nature\} 414:541-543
%RETRACTED?

\bibitem{label7} J. Doebley (1990)
Molecular Evidence for Gene Flow among {\em Zea\} Species.
{\em BioScience\} 40:443-448

\bibitem{label8} J. F. Doebley (1984)
Maize Introgression into Teosinte - A Reappraisal.
{\em Annals of the Missouri Botanical Garden\} 71:1100-1113

\bibitem{label9} J. Doebley (1990)
Molecular Evidence and the Evolution of Maize.
{\em Economic Botany\} 44:6-27
%a review paper

\bibitem{label10} J. Doebley, M. M. Goodman, C. W. Stuber (1987)
Patterns of isozyme variation between maize and Mexican annual teosinte.
{\em Economic Botany\} 41:234-246

\bibitem{label11} B. M. Baltazar, J. de J. Sanchez-Gonzalez (2005)
Pollination between maize and teosinte: an important determinant of gene flow in Mexico
{\em Theoretical Applied Genetics\} 110:519-526
%special character for the Sanchez name, plus how to abbreviate "de"

\bibitem{label12} M. L. Warburton, G. Wilkes, S. Taba, A. Charcosset, C. Mir, F. Dumas, D. Madur, S. Dreisigacker, C. Bedoya, B. M. Prasanna, C. X. Xie, S. Hearne, J. Franco (2011)
Gene flow among different teosinte taxa and into the domesticated maize gene pool.
{\em Genetic Resources and Crop Evolution\} 58:1243-1261

\bibitem{label13} C. Jiang, G. O. Edmeades, I. Armstead, H. R. Lafitte, M. D. Hayward, D. Hoisington (1999)
Genetic analysis of adaptation of adaptation differences between highland and lowland tropical maize using molecular markers.
{\em Theoretical Applied Genetics\} 99:1106-1119

\bibitem{label14} E. Bitocchi, L. Nanni, M. Rossi, D. Rau, E. Bellucci, A. Giardini, A. Buonamici, G. G. Vendramin, R. Papa (2009)
Introgression from modern hybrid varieties into landrace populations of maize ({\em Zea mays\} ssp. {\em mays\} L.) in central Italy.
{\em Molecular Ecology\} 18:603-621

\bibitem{label15} M. L. Arnold, N. H. Martin (2009)
Adaptation by introgression.
{\em Journal of Biology\} 8:82
%mini review of introgression

\bibitem{label16} N. C. Ellstrand, C. Garner, S. Hegde, R. Guadagnuolo, L. Blancas (2007)
Spontaneous hybridization between Maize and Teosinte.
{\em Journal of Heredity\} 98:183-187

\bibitem{label17} M. B. Hufford, P. Lubinsky, T. Pyhajarvi, M. T. Devengenzo, N. C. Ellstrand, J. Ross-Ibarra (2013)
The genomic signature of crop-wild introgression in maize.
{\em PLOS Genetics\} 9:
%special characters, page numbers?

\bibitem{label18} Y. Matsuoka, Y. Vigouroux, M. M. Goodman, J. Sanchez G., E. Buckler, J. Doebley (2002)
A single domestication for maize shown by multilocus microsatellite genotyping.
{\em PNAS\} 99:6080-6084

\bibitem{label19} J. van Heerwaarden, J. Doebley, W. H. Briggs, J. C. Glaubitz, M. M. Goodman, J. de Jesus Sanchez Gonzalez, J. Ross-Ibarra (2010)
Genetic signals of origin, spread, and introgression in a large sample of maize landraces.
{\em PNAS\} 108:1088-1092
%Gonzalez, "van"

\bibitem{label20} H. G. Wilkes (1970)
Teosinte introgression in the maize of the Nobogame Valley
{\em Botanical Museum Leaflets\} 22:297-311

\bibitem{label21} M. L. Arnold (2004)
Transfer and origin of adaptations through natural hybridization: Were Anderson and Stebbins Right?
{\em The Plant Cell\} 16:562-570
%introgression review

\bibitem{label22} C. N. Stewart Jr., M. D. Halfhill, S. Warwick (2003)
Transgene introgression from genetically modified crops to their wild relatives.
{\em Nature\} 4:806-817
%is nature reviews the same thing as nature?
%good review of introgression with crops
%Table 1 shows crops and their progenitor species and main wild relatives

\bibitem{label23} L. H. Rieseberg, J. F. Wendel (1993)
Hybrid zones and the evolutionary process.
{\em Journal of Evolutionary Biology\} 7:631-634

\bibitem{label24} E. Nevo, G. Chen (2010)
Drought and salt tolerances in wild relatives for wheat and barley improvement
{\em Plant, Cell \& Environment\} 33:670-685

\bibitem{label25} E. Autrique, R. P. Singh, S. D. Tanksley, M. E. Sorrells (1995)
Molecular markers for four lef rust resistance genes introgressed into wheat from wild relatives
{\em Genome\} 38:75-83

\bibitem{label26} S. G. Hegde, J. G. Waines (2004)
Hybridization and introgression between bread wheat and wild and weedy relatives in North America.
{\em Crop Science\} 44:1145-1155

\bibitem{label27} M. A. Chapman, J. M. Burke (2006)
Letting the gene out of the bottle: The population genetics of genetically modified crops.
{\em New Phytologist\} 170:429-443

\bibitem{label28} S. B. Broyles (2002)
Hybrid bridges to gene flow: A case study in milkweeds ({\em Asclepias\}).
{\em Evolution\} 56:1943-1953


\bibitem{label29} S. W. Hussain, W. M. Williams (1997)
Development of a fertile genetic bridge between {\em Trifolum ambiguum\} M. Bieb. and {\em T. repens\} L.
{\em Theoretical Applied Genetics\} 95:678-690

\bibitem{label30} M. von Korff, H. Wang, J. Leon, K. Pillen (2006)
AB-QTL analysis in spring barley: II. Detection of favourable exotic alleles for agronomic traits introgressed from wild barley ({\em H. vulgare\} ssp. {\em spontaneum\}).
{\em Theoretical Applied Genetics\} 112:1221-1231
%special characters

\bibitem{label31} D. Zohary, J. R. Harlan, A. Vardi (1969)
The wild diploid progenitors of wheat and their breeding value.
{\em Euphytica\} 18:58-65

\bibitem{label32} [[A huge fricken list of names]] (2013)
Hybridization and speciation.
{\em Journal of Evolutionary Biology\} 26:229-246

\bibitem{label32} K. Fukunaga, J. Hill, Y. Virouroux, Y. Matsuoka, J. Sanchez G., K. Liu, E. S. Buckler, J. Doebley (2005)
Genetic diversity and Population Structure of Teosinte.
{\em Genetics\} 169:2241-2254

\bibitem{label33} M. Reagon, A. A. Snow (2006)
Cultivated {\em Helianthus annus\} (Asteraceae) volunteers as a genetic "bridge" to weedy sunflower populations in North America.
{\em Ammerican Journal of Botany\} 93:127-133

\bibitem{label34} J. A. Able, P. Langridge, A. S. Milligan (2006)
Capturing diversity in the cereals: many options but little promiscuity.
{\em TRENDS in Plant Science\} 12:1360-1385

\end{document}
