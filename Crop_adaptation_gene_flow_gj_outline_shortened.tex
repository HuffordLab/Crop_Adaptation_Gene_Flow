\documentclass[11pt]{article}
\usepackage{color}
\usepackage{authblk}%allows footnote format for authors
\usepackage[letterpaper, margin=1in]{geometry} %package that allows changes in margins and header/footers
\usepackage{natbib}
\bibliographystyle{ieeetr}
\newcommand{\mbh}[1]{\textcolor{green}{ \emph{\scriptsize  #1}} } %creating command for Matt's comments
\newcommand{\lwang}[1]{\textcolor{red}{ \emph{\scriptsize  #1}} } %creating command for Li's comments

\title{Crop adaptation through gene flow}

\begin{document}

\maketitle

\section*{Outline}

\subsection*{Introduction}

 "Whenever sympatric populations of crops and their wild relatives show a similar trait, there will be several possible explanations: (1) retention of an ancestral trait, (2) convergent or parallel evolution, and (3) introgression.  Perhaps even more difficult is establishing the direction of introgression; does gene flow go from crop into wild species, the reverse, or in both directions?" and "For this reason, introgression may be best studied between crops and their more distant wild relatives where the probability of finding taxon-specific genetic markers is increased." \cite{doebley1990molecular}

\subsection*{Maize}

\subsection*{wheat}

The convoluted domestication of wheat is marked by many instances of hybridization between morphologically-distinct species within a restrained geographical range.
Throughout wheat domestication, there has been gene flow between species and between wild and domesticated wheats.
Emmer wheat, for example, has high genetic diversity, likely due to gene flow from wild emmer \cite{luo2007structure, dvorak2006molecular}\.
The a, b, c, and f alleles of mildew resistance gene Pm3 may have been introgressed into wheat from wild emmer wheat shortly after domestication (although the e, d, and g alleles were probably formed by de novo mutations) \cite{TPJ:TPJ2772}\.
Furthermore, wild wheat relatives are can be found in a much broader swath than the domesication center in the Fertile Crescent \cite{CWR}\.

\subsection*{barley}

Wild-domesticate breeding experiments have shown that wild barleys have alleles for several important agronomic phenotypes, including brittleness, flowering time, plant height, lodging, and yield,  \cite{von2006ab, handley1994chromosome}\.
Although the conditions of barley domestication would seem to allow (if not promote) natural adaptive introgression between barley and its wild relatives, there is little evidence of this genetic interaction at present.
Barley/\emph{spontaneum} hybrids are fertile, and morphologically intermediate (putatively hybrid) barleys are found when wild and domesticated barleys are grown in sympatry, but hybrids of other wild relatives generally exhibit greatly diminished fertility \cite{ellstrand2003dangerous, harlan1995living}\.

\subsection*{rice}

Several resistance genes (grassy stunt virus, bacterial blight, brown planthopper, blast) have been introgressed from wild relatives into \emph{O. sativa} by researchers \cite{brar1997alien, khush1974inheritance}.
Beyond investigative experiments, gene flow from wild relatives has been used to produce agronomic rice varieties.
Yatsen No. 1, for example, showed resistance to pests and diseases and adapted well to environmental conditions \cite{ting1933wild}\.
Several lines were derived from Yatsen No. 1, and went on to be utilized extensively in parts of China.

\subsection*{potato}

Introgressive hybridization is widespread in potatoes \cite{grun1990evolution}\.
Andean potatoes exhibit high ecological versatility, due in part either to alleleic diversity in polyploids or introgression of desirable alleles from wild relatives in diploids \cite{zimmerer1998ecogeography}\.
Cultivated diploid potatoes have been shown introgressed with genes from \emph{S. sparsipilum} \cite{rabinowitz1990high}\ and \emph{S. megistacrolobum} \cite{johns1987relationships, huaman1980solanum}.
Drought- and frost-resistance genes introgressed from wild relatives expediated the spread of cultivated potatoes into the central Andean Altiplano \cite{johns1986ongoing, hawkes1962origin, schmiediche1980breeding}\.
Resistance genes have been experimentally introgressed from wild poptato relatives (\emph{S. tuberosum subsp. andigena}, \cite{van1999tight}, \emph{S. bulbocastanum}, \cite{van2003ancient}) into cultivated potato, although this process is hindered by compatability issues including ploidy levels and Endosperm Balance Number \cite{johnston1980significance}.

\subsection*{Sugarcane}

The history of sugarcane is marked with phases of concerted interspecific breeding efforts, primarily but not exclusively between S. officinarum and S. spontaneum \cite{roach1989origin}\.  

From \cite{ellstrand1999gene}\:
33 - daniels1987taxonomy
    "genetically based traits" evidence for hybrid origin of "many cultivars"
    "chromosomal evidence" supports "spontaneous intergeneric hybridization" origin of "particular accessions of wild sugarcane relatives"
    Erianthus maximus might be a hybrid of S. officinarium and Miscanthus (a wild relative genus), based on morphological and chromosomal evidence.
104 - roach1995sugar
    "spontaneous hybridization between cultivated and wild Saccharum in Australasia and islands of the Indian Ocean" in origin of "many cultivars"

\subsection*{Cassava}

Although cultivated cassava (Manihot esculenta) is interfertile with some of its wild relatives (the putative progenitor species M. flabellifolia, for example, \cite{roa1997aflp}\), comparisons of single-copy nuclear gene G3pdh sequences provide little evidence to suggest that natural introgressive hybridization occurs, and if it does, it would appear to be in the direction of cultivar to wild relative \cite{olsen1999evidence}\.
Clonal propogation and allopatry are listed among the barriers to gene flow between cassava and its wild relatives [[but see tovar2015diversity for discussion on elevated genetic diversity in M. esculenta, gene flow between species, occasional sexual reproduction, and trade between farmers]].
Breeders have introgressed genes for protein and disease resistance genes into cassava from M. flabellifolia \cite{akinbo2008introgression}\.

\subsection{Sweet Potato}

Sweet potato (Ipomoea batatas) likely arose from multiple domestications, in Central America and in South America, which were secondarily hybridized upon introduction \cite{roullier2013disentangling}.  [[see austin1988taxonomy for a summary of the genus Ipomoea, wild relatives and such.]]

Olive

Morphological similarities indicate that wild, weedy, and cultivated olives form an interfertile species complex that sprawls across the Mediterranean, and that outbreeding with these wild and weedy relatives has contributed to olive's genetic diversity, but it has not been shown that locally-adapted genes have been introgressed into the domesticated olive \cite{zohary1975beginnings}\.



















\subsection*{Sorghum}
Anthropological evidence points to a domestication event of sorghum (\emph{Sorghum bicolor subsp. bicolor}) some 5-6,000 years ago in Ethiopian Sudan \cite{smith2000sorghum}\.
However, comparative isozyme analyses \cite{shechter1975comparative}\ have suggested a possible secondary domestication event (sorghum race kafir) in the southeastern Savanna.
The progenitor species was \emph{S. arundinaceum} \cite{doggett1988sorghum, harlan1971toward}\.
The genus Sorghum contains 25 species, but these are split between five subgenera: Eu-Sorghum, Chaetosorghum, Heterosorghum, Para-Sorghum, and Stiposorghum, and it is Eu-Sorghum that includes all domesticated, weedy, and closely-related wild relatives \cite{USDAARS2007, garber1950cytotaxonomic}.
Within Eu-Sorghum, there are currently nine races, five cultivated (bicolor, guinea, kafir, caudatum, and durra) and four wild (arundinaceum, virgatum, aethiopicum, and verticilliflorum) \cite{smith2000sorghum}\.
Members of \emph{Sorghum bicolor} are interfertile, though outcrossing rates and fertility can be low due to ploidy incompatability \cite{doggett1988sorghum, arriola1996crop}.

\cite{po1982sorghum}\ states that natural introgression has been an important force in sorghum evolution.
Domesticate-weedy-wild complexes are common when sorghum cultivars are grown in sympatry with wild members of the Eu-Sorghum subgenus \cite{de1978systematics, doggett1968disruptive, baker1972human}\.
Rates of outcrossing within sorghum range from 0-30% \cite{doggett1988sorghum}\.
Natural introgression between wild and domesticated sorghum has been documented in both directions \cite{kuhlman2006interspecific, aldrich1992restriction, aldrich1992patterns, doggett1988sorghum, baker1972migrations}\.
These hybrids express diminished fertility.
Members of \emph{Sorghum bicolor} have adapted to a broad range of altitudinal, precipitation, and temperature clines across Africa and around the world \cite{po1982sorghum, CWR}\.
Breeding efforts have focused on incorporating exotic germplasm (wild sorghums as well as relatives from other genera \cite{de1976cytogenetics} for its adaptations to biotic and abiotic stresses \cite{reddy2006current, po1982sorghum, johnson1979breeding}.
Resistance to greenbug has been introgressed into cultivated sorghum through modern breeding efforts \cite{johnson1979breeding}.















/subsection*{Soybean}
Soybeans (Glycine max) were domesicated in southern China \cite{guo2010single}\.
Soybeans (Glycine max) primarily self-pollinate, outcrossing at a rate of 1-8% \cite{ray2003soybean}\.
Soybeans readily cross with their wild progenitor, G. soja \cite{singh1988genomic}\.
Plants of intermediate morphology appear around fields when S. soja is also present, indicating spontaneous hybridization \cite{kwon1972studies}\.
The accession "G. gracilis" shows allelic contribution from both G. max and G. soja \cite{keim1989restriction}\.

\bibliography{bib_gj.bib}
\end{document}
