\documentclass[11pt]{article}
\usepackage{color}
\usepackage{authblk}%allows footnote format for authors
\usepackage[letterpaper, margin=1in]{geometry} %package that allows changes in margins and header/footers

\newcommand{\mbh}[1]{\textcolor{green}{ \emph{\scriptsize  #1}} } %creating command for Matt's comments
\newcommand{\lwang}[1]{\textcolor{red}{ \emph{\scriptsize  #1}} } %creating command for Li's comments

\title{Crop adaptation through gene flow}

\begin{document}

\maketitle

\section*{Outline}

\subsection*{Introduction}

 "Whenever sympatric populations of crops and their wild relatives show a similar trait, there will be several possible explanations: (1) retention of an ancestral trait, (2) convergent or parallel evolution, and (3) introgression.  Perhaps even more difficult is establishing the direction of introgression; does gene flow go from crop into wild species, the reverse, or in both directions?" and "For this reason, introgression may be best studied between crops and their more distant wild relatives where the probability of finding taxon-specific genetic markers is increased." \cite{doebley1990molecular}

\subsection*{Maize}


Supplement Li's work with the following ideas:


Maize is a classical model in crop species to study the issue of adaptive introgresssion from wild relatives, as it still grows in sympatry with many of its wild relatives \cite{hufford2013}.

Maize was domesticated about 10,000 years ago in Mexico \cite{smith1997initial} \cite{hufford2012comparative} in the Balsas River valley \cite {matsuoka2002single} from teosinte.

Mexican farmers have long been aware of maize-teosinte hybrids, and in some places, such as the Nobogame valley, farmers have observed desirable traits passed to their crops through the hybrid plants \cite{wilkes1977hybridization} \cite{lumholtz1902unknown} \cite{wilkes1970teosinte}.

In particular, \emph{Zea mays} ssp. \emph{mexicana} (referred to \emph{mexicana} hereafter) and Mexican highland maize have become a well-studied pair (although ssp. \emp{parvaglumis} has also been shown to hybridize with maize, \cite{wilkes1977hybridization}).

Introgression between \emph{mexicana} and Mexican highland maize has been reported based on evidence from both morphological data \cite{wilkes1977, lauter2004, doebley1984} and molecular analysis \cite{matsuoka2002, vanHeerwaarden2011, doebley1987, warburton2011, fukunaga2005}. [[Li's and my bib tags might be in conflict, e.g. wilkes1977 is likely my wilkes1977hybridization.]]
.
 Doebley (\cite{doebley1987patterns}) found that north-central high-elevation Mexican landraces (Apachito, Arrocillo Amarillo, Azul, Celaya, Chaloqueno, Conico, Conico Norteno, Gordo, Harinoso de Ocho Occidentales)[[special characters]] show evidence of introgression of two alleles from \emph{mexicana}, and suspected that they might have fitness advantages for adapting to highland environmental conditions.

Population structure analyses by \cite{matsuoka2002single} estimate that \emph{mexicana} gene flow has contributed to the gene pools of the Cacahuacintle, Palomero de Jalisco, and Palomero Taloqueno landraces, but also into (non-landrace) Mexican maize at the same elevation.

However, \cite{hufford2013} for the first time revealed the evidence of adaptive introgression from \emph{mexicana} to Mexican highland maize.
The authors identified nine genomic regions, which showed evidence of introgression from \emph{mexicana} to maize in both the HAPMIX and the linkage model of STRUCTURE analyses with over seven sympatric population pairs among the total nine pairs sampled. 
Among the nine regions, three spans the centromeres of chromosomes 5, 6, and 10, and one is located in the inversion polymorphism on chromosome 4, suggesting a significant role of genome structures restricting recombination in adaptive introgression.
By further characterizing the nine introgression regions, it is found that most regions contain long tracts of zero diversity, enriched with QTL linked with anthocyanin content and leaf macrohairs \cite{lauter2004} and over-represented with the SNPs demonstrating high association with temperature seasonality.
Growth chamber experiments with maize populations with introgression from \emph{mexicana} on chromosome 4 (associated with QTL controlling pigment density and macrohairs) and 9 (overlapped with QTL for macrohairs) exhibited more macrohairs and greater pigmentation under the highland environmental settings than the populations with absence of introgression from \emph{mexicana}.

Gene flow between \emph{mexicana} and maize is low enough that only fitness-neutral or advantageous alleles are likely to be passed \cite{slatkin1987gene}.

From \cite{baltazar2005pollination}
Gene flow is probably greater from teosinte to maize.
Certain morphological and flower timing characterists promote genetic isolation between maize and teosinte and slow rates of introgression.
Gene flow from maize to teosinte occurs most easily when teosinte pollinates maize
[[what paper can i cite to mention the gene tb1 and its role in hybridization, viz a viz restricting directionality of teosinte pollen into maize

From \cite{jiang1999genetic}, {bitocchi2009introgression}
Maize-maize introgression
Research on introgression between highland and lowland maize shows adaptive traits being transferred

\cite{bitocchi2009introgression} has a good review of the history of domestication, introduction to Europe, and further development.
\cite{hammer1987collection}\ cites an example in Italy of one domesticated maize adapting to coastal environment by introgressions from another domesticated maize.

\subsection*{wheat}

The first domesticated wheat was einkorn wheat (Triticum monococcum), domesticated from wild einkorn (T. m. boeoticum) in or around the Karacadag mountains of southeast Turkey \cite{heun1997site, harlan1966distribution, ozkan2002aflp}\.
Subsequently, emmer wheat (Triticum dicoccum) was domesticated from hybrids of diverse populations of wild emmer wheat (Triticum dioccoides), also in the Fertile Crescent \cite{lev2000cradle, civavn2013reticulated, luo2007structure, ozkan2002aflp}.
Two distinct taxa of T. dioccoides exist (a western one and a central-eastern one), but the central-eastern one was the progenitor species from which modern wheat was domesticated \cite{ozkan2005reconsideration}
The tetraploid durum wheat (Triticum turgidum) is the result of hybridization between red wild einkorn wheat (T. urartu) and wild goatgrass (Aegilops speltoides) \cite{mcfadden1946origin, petersen2006phylogenetic}\, and a further hybridization with Tausch's goatgrass (Aegilops tauschii) produced the allohexaploid common bread wheat (T. aestivum), the most agronomically important race of wheat \cite{salamini2002genetics, hancock2012plant, dvorak2006molecular}\.
This hexaploidization event probably occured twice, once in southeast Turkey/northern Syria and, more recently, in Iran \cite{giles2006gludy}\.
Each of the three genomes that comprise bread wheat shows genomic signatures of ancestral hybridizations as well \cite{marcussen2014ancient}\.

The progressive domestication of wheat has involved many introgressive acts between geographically dispersed lineages, evinced by identification of genomes' various sources in wheats with elevated ploidy levels.

The convoluted domestication of wheat is beset by many instances of hybridization between morphologically-distinct species within a restrained geographical range.
Throughout wheat domestication, there has been gene flow between species and between wild and domesticated wheats.
Emmer wheat, for example, has high genetic diversity, likely due to gene flow from wild emmer \cite{luo2007structure, dvorak2006molecular}\.
The a, b, c, and f alleles of mildew resistance gene Pm3 may have been introgressed into wheat from wild emmer wheat shortly after domestication (although the e, d, and g alleles were probably formed by de novo mutations) \cite{TPJ:TPJ2772}.
Furthermore, wild wheat relatives are can be found in a much broader swath than the domesication center in the Fertile Crescent \cite{CWR}\.
Modern breeding efforts between wild and domesticated wheat have shown some success.
For example, some researchers have experimentally bred resistance traits into durum wheat (stem rust resistance from wild einkorn, \cite{gerechter1971transfer}\, and powdery mildew resistance from T. turgidum var. dicoccoides, \cite{blanco2008molecular}\.

\subsection*{barley}

Domesticated and wild barleys belong to the same species, Hordeum vulgare, and are biologically capable of producing viable offpsring via hybridizaion \cite{von1995ecographical}.
Barley (Hordeum vulgare subsp. vulgare) is believed to have been domesticated at least twice, once from wild subsp. spontaneum in the Fertile Crescent and once by subsp. spontaneum var. agriocrithon in Tibet roughly 10,000 years ago \cite{takahashi1955origin, badr2000origin, oka2012origin, azhaguvel2007phylogenetic, haberer2015barley}\, but many of the details of barley domestication are still disputed.

Variety agriocrithon (H. vulgare ssp. vulgare f. agriocrithon) is genetically diverse, and is found throughout much of the range of Barley.
Some have suggested that agriocrithon may be the progenitor of six-rowed barley, the product of a hybridization between eastern and western cultivated barley, or a wild-domesticate hybrid \cite{staudt1961origin, zohary1959hordeum, murphy1982origin}\, but \cite{azhaguvel2007phylogenetic}\ dispute these theories, suggesting instead that it arose from ssp. spontaneum, perhaps more than once (in Israel as well as in Tibet).
Presently, the distribution of spontaneum consists of the Mediterranean, the Middle-East, and West-Central Asia, while other barley wild relatives inhabit Asiatic regions (including Tibet) more broadly \cite{nevo2010drought, harlan1995living, CWR}\.
These wild relatives inhabit regions spanning such abiotic clines as temperature, precipitation, soil type, and altitude, as well as biotic clines \cite{nevo2010drought}\.

The barley domestication process has reduced the number of alleles in the domesticate to only 40% of that found in wild barley, though there remains a great deal of phenotypic diversity among the wild barleys \cite{ellis2000wild}\.
Wild-domesticate breeding experiments have shown that wild barleys have alleles for several important agronomic phenotypes, including brittleness, flowering time, plant height, lodging, and yield,  \cite{von2006ab, handley1994chromosome}\.
Although the conditions of barley domestication would seem to allow (if not promote) natural adaptive introgression between barley and its wild relatives, there is little evidence of this genetic interaction at present.
Owing to this sparcity of evidence is the low rate of outcrossing of spontaneum, estimated by \cite{brown1978outcrossing} to be 1.6% in Israeli populations.
There has been little genetic investigation into spontaneous barley/spontaneum hybrids \cite{ellstrand2003dangerous}.
Barley/H. spontaneum hybrids are fertile, and morphologically intermediate (putatively hybrid) barleys are found when wild and domesticated barleys are grown in sympatry, but hybrids of other wild relatives generally exhibit greatly diminished fertility \cite{ellstrand2003dangerous, harlan1995living}\.
Even when the two are not grown immediately adjacent to one another, introgression from wild to domesticate has been shown to happen over distances of more than a kilometer \cite{hillman2001new}\.

\subsection*{sunflower}

The common sunflower (Helianthus annuus) shows evidence of domestication in eastern United States \cite{harter2004origin, wills2006chloroplast}\, with additional evidence of a possible second origin of domestication in Mexico \cite{lentz2008sunflower}\.
Helianthus is a versatile species, capable of adapting to a wide range of environments across the Americas, Europe, Asia, and Australia \cite{kane2008genetics}\.
[[This versatility may be due in part either to phenotypic plasticity or genetic adaptability \cite{maron2004rapid}\.]] This sentence may not help much, actually.  We're betting on genetic adaptability via introg hyb
H. annuus adapted to the environment in Texas by hybridizing with H. debilis ssp. cucumerifolius, gaining advantageous alleles (these hybrids are now called H. annuus ssp. texanus, \cite{kim1999genetic, heiser1951hybridization, rieseberg1999hybrid, rieseberg1990helianthus}.
H. annuus lives in sympatry with wild relatives like H. petiolaris and H. bolanderi and forms  stable hybrid populations \cite{schwarzbach2002likely, rieseberg1988molecular, welch2002patterns}\.
H. annuus shows signs of persistent introgressive hybridization with H. petiolaris with evidence of positive selection driving some of the genetic differentiation between the two species \cite{yatabe2007rampant}\.
Wild sunflowers are locally adapted, and weedy hybrid populations share these adaptations \cite{kane2008genetics}\.
Weediness is a trait that has likely evolved many times \cite{kane2008genetics}\.

Helianthus has several genes for downy mildew resistance.
Each imparts resistance to one or more races of P. halstedii, one of the most agronomically important diseases in sunflower cultivation \cite{cohen1973factors}\.
Some of these downy mildew resistence genes were found in wild relatives (including H. argophyllus, H. tuberosus, and H. praecox) and have been successfully bred into modern H. annuus \cite{miller1991inheritance}\.
PlArg, an allele found in wild silverleaf sunflowers (Helianthus argophyllus, inbred line Arg1575-2), confers resistance to all known (20 or more) races of downey mildew \cite{dussle2004pl}\, while others (Pl1-Pl11) are effective for one or more types \cite{rahim2002inheritance}\.
Silverleaf sunflower has also been the focus of drought resistance breeding efforts \cite{saucă2010introgression}\ and Phomopsis resistance breeding efforts \cite{besnard1997specifying}\.

\subsection*{rice}

There are two main cultigens of rice; Asian rice (Oryza sativa) and African rice (O. glaberrima), which are easily distinguish by ligule length, number of secondary panicle branches, panicle axis thickness, and differences in life cycles \cite{oka2012origin}.
The centers of rice domestication are not known with complete confidence, but genetic and archaeobotanical evidence points towards both the Yangzee Basin in China and the Ganges plains in India for O. sativa, 8,200-13,500 years ago, from wild O. rufipogon  \cite{oka2012origin, fuller2010consilience, ricepedia, molina2011molecular}\ or "the Asian form of O. perennis complex" \cite{oka2012origin}\ and the Upper Niger River delta in Mali, Africa for O. glaberrima, 2-3,000 years ago, from a wild ancestor, perhaps O. barthii \cite{ricepedia}\ or O. breviligulata \cite{oka2012origin}\.
Asian rice has two main subspecies, Indica (with subpopulations indica and aus) and Japonica (with subpopulations temperate japonica, tropical japonica known as javanica, and aromatic) \cite{chang2003origin, glaszmann1987isozymes, ricepedia}.
Indica and Japonica are likely the result of independent domestications from separate O. rufipogon populations in India/Indochina and southern China, respectively \cite{londo2006phylogeography}\.
These subspecies display adaptations to the environmental coniditions corresponding to the differentiated geographical locations they inhabit \cite{khush2003classifying}\.

Both Asian and African rice naturally hybridize with other domesticated subspecies and with wild relatives (of which there are about twenty, \cite{ricepedia}\), and introgression is common \cite{oka2012origin, second1982origin, zhao2010genomic}\.
The greater genetic diversity within O. sativa is likely due to introgression with wild relatives both during domestication and upon the dispersal of O. sativa into new environments and sympatry with new relatives \cite{second1982origin}.
Several resistance genes (grassy stunt virus, bacterial blight, brown planthopper, blast) have been introgressed from wild relatives into O. sativa L. by researchers \cite{brar1997alien, khush1974inheritance}.
Beyond investigative experiments, gene flow from wild relatives has been used to produce agronomic rice varieties.
Yatsen No. 1, for example, showed resistance to pests and diseases and adapted well to environmental conditions \cite{ting1933wild}\.
Several lines were derived from Yatsen No. 1, and went on to be utilized extensively in parts of China.

\subsection*{rye}

Rye domestication has recieved comparatively little academic interest.
Although there is little evidence for or against the role of natural introgression in rye domestication, rye (Secale cereale) has wild relatives outside its center of domestication in or around the Fertile Crescent \cite{vavilov1928geographical}\, and the species contains domesticated, weedy, and wild rye subspecies \cite{khush1961cytogenetic}\.

\subsection*{potato}

Modern potato (Solanum tuberosum) is believed to have been domesticated in southern Peru in sympatry with a multitude of wild relatives about 6000 years ago, although the exact location and formal classification and phylogenetic relationships between these taxa have long been disputed \cite{huaman2002reclassification, spooner2005single, pickersgill1977origins, hawkes1988evolution}\.
The northern members of the polyphyletic S. brevicaule wild potato complex have been identified as likely progenitors of modern potatoes \cite{correll1962potato}, but determination of a single progenitor species is unlikely, either because widespread gene flow in the complex will mask signs of this phylogeny or because s. tuberosum has a polyphyletic origin.
Also, these northern members of the S. breviaule complex are not clearly defined, and may in fact be one singular species (in which case the species name would be S. bukasovii, \cite{spooner2005single}\).

Although potatoes are usually propogated clonally, farmers also promote sexual hybridization at times to improve disease resistance and develop new cultivars \cite{quiros1992increase}\.
Farmers continue to grow potatoes in close proximity to wild relatives, resulting in domesticate-weedy-wild hybrid complexes which promote introgressive hybridizatiion \cite{rabinowitz1990high, johns1987relationships, linder1987diversity}\.
These complexes, combined with a diverse range of biotic and environmental selective pressures and local farming practices (human-mediated migration, isolated farmsteads in fertile valleys, clonal propogation, and intentional maintenance of a variety of landraces), have fostered expansion of genetic diversity within potatoes subsequent to domestication \cite{brush1995potato}\.
However, as farmers tend to abandon fields after being used for potato cultivation, it is less likely that hybrids have an opportunity to form stable populations for maintained introgressive gene flow into the domesticated potato gene pool \cite{brush1995potato}\.
Also, natural selection may be subverted by farmer preference and artificial selection \cite{brush1981dynamics}\.

Introgressive hybridization is widespread in potatoes \cite{grun1990evolution}\.
The various cultivars of Andean potatoes are interfertile, forming one large plastic gene pool \cite{quiros1992increase}\.
Andean potatoes exhibit high ecological versatility, due in part either to alleleic diversity in polyploids or introgression of desirable alleles from wild relatives in diploids \cite{zimmerer1998ecogeography}\.
Cultivated diploid potatoes have been shown introgressed with genes from S. sparsipilum \cite{rabinowitz1990high}\ and S. megistacrolobum \cite{johns1987relationships, huaman1980solanum}.
Drought- and frost-resistance genes introgressed from wild relatives expediated the spread of cultivated potatoes into the Central Andean Altiplano \cite{johns1986ongoing, hawkes1962origin, schmiediche1980breeding}\.
Resistance genes have been experimentally introgressed from wild poptato relatives (S. tuberosum subsp. andigena, \cite{van1999tight}, S. bulbocastanum, \cite{van2003ancient}) into cultivated potato, although this process is hindered by compatability issues including ploidy levels and Endosperm Balance Number \cite{johnston1980significance}.

\subsection*{tomato}

The domestication center for tomato is not known with certainty, but \cite{sims1979history} quotes Dr. Charles Rick (University of California, August 1978) stating that tomato was domesticated from Lycopersicon esculentum v. cerasiformae [[solanum lycopersicum cerasiforme]] in the New World (most likely Mexico) [[http://www.landscapeimagery.com/tomato.html]].
\cite{robertson2007genetic, bai2007domestication} and [[http://www.landscapeimagery.com/tomato.html]] agrees.
\cite{nesbitt2002comparative}, however, conclude that v. cerasiformae is not the direct ancestor of the domesticated tomato, but rather a hybrid between wild tomatoes L. esculentum and L. pimpinellifolium.
Widescale recent introgression between wild relatives confound attempts at elucidating phylogenies.

Since the work of Charles Rick in 1940s and 50s \cite{rick1953novel}\, breeding efforts have focused on incorporating and utilizing the relatively diverse genetic germplasm of wild tomato relatives \cite{rick1988tomato, miller1990rflp, rick1982potential}\.
Wild tomato relatives have been shown to harbor resistance genes for more than 40 agronomically important diseases, and breeders are working to introgress them into cultivated tomato \cite{rick1995utilization}\.
\cite{hanson2000mapping} mapped a tomato yellow leaf curl virus resistance gene in an accession of L. hirsutum f. glabratum \cite{banerjee1990transfer}\ to an introgression on chromosome 11, which originated from the wild relative L. hirsutum.
Similar resistance traits has also been shown to have been imparted from L. chilense, L. pimpinellifolium, and L. cheesmanii \cite{hanson2000mapping}\.

\subsection*{Common Bean}

Common bean (Phaseolus vulgaris) was domesticated from a wild common bean at least twice, once in the Lerma-Santiago Basin of Mesoamerica and once in the southern Andes \cite{kwak2009putative}\.
The domesticated common bean progenitor species (P. vulgaris var. aborigineus) and other wild relative species can be found across the breadth of this region and beyond, from Mexico to Argentina, inhabiting a broad range of environments \cite{debouck1988implications, CWR}\.
Because common bean is generally self-pollinating (outcrossing at an estimated rate under 3%, \cite{ramalho2006}\), rates of gene flow between it and wild relatives are expected to be lower.
These expectations are supported by evidence provided by \cite{papa2003asymmetry}\.
However, hybrid populations can still be found in sympatry with wild and domesticated common bean, forming domesticate-hybrid-weedy complexes.

Some evidence for introgression exists in the direction of domesitcated to wild common bean, leading to increased seed size, an adaptation that permitted their spread to higher elevations with cooler climates \cite{debouck1993genetic}\.
Wild common beans with this introgression are also occasionally harvested along with cultivated common bean and consumed by subsistence farmers \cite{papa2003asymmetry}\.
\cite{kwak2009structure}\ also found via STRUCTURE analysis that certain Andean wild common bean accessions were possibly the result of hybridizations with domesticated common bean.
The disparity in gene flow directionality may be due in part to the large population size of common beans in a cultivated field relative to the population size of wild common beans within pollen dispersal range (a distance that may be quite small, \cite{papa2003asymmetry}\).
Farmer practices may also be diminishing introgressive gene flow.
Though some hybrids show heterosis, others (especially F1 hybrids) do not, expressing smaller plant size, smaller seed size, and increased mortality \cite{gutierrez1985heterosis, paredes1995extensive}\.
Because F1 hybrids are intermediate in seed size and color, farmers may effectively identify and avoid hybrid stock.

Though limited, there is also some evidence of natural introgression of wild genes into domesticated common bean.
\cite{papa2003asymmetry}\ was unable to determine how much of the genetic variability of domesticated common bean landraces originated from weedy or wild types, but point to the allele for 'L' phaseolin as a possible example of an allele that has been introgressed from wild common beans into domesticated.
Experimentally, common leaf blight and white mold resistances have been bred into common bean from scarlet runner bean (P. coccineus) \cite{park1987transfer, schwartz2006inheritance}\, as was resistance to weevils (Zabrotes subfasciatus and Acanthoscelides obtectus) from Mexican wild bean species \cite{kornegay1991inheritance}\, but it may take careful, larger-scale breeding programs to exploit heterotic hybrids for agronomic gain \cite{paredes1995extensive}\.

\subsection*{Sorghum}
Anthropological evidence points to a domestication event of sorghum (Sorghum bicolor subsp. bicolor) some 5-6,000 years ago in Ethiopian Sudan \cite{smith2000sorghum}\.
However, comparative isozyme analyses \cite{shechter1975comparative}\ have suggested a possible secondary domestication event (sorghum race kafir) in the southeastern Savanna.
The progenitor species was S. arundinaceum \cite{doggett1988sorghum, harlan1971toward}\.
The genus Sorghum contains 25 species, but these are split between five subgenera: Eu-Sorghum, Chaetosorghum, Heterosorghum, Para-Sorghum, and Stiposorghum, and it is Eu-Sorghum that includes all domesticated, weedy, and closely-related wild relatives \cite{USDAARS2007, garber1950cytotaxonomic}.
Within Eu-Sorghum, there are currently five cultivated (bicolor, guinea, kafir, caudatum, and durra) and four wild (arundinaceum, virgatum, aethiopicum, and verticilliflorum) \cite{smith2000sorghum}\.
Members of Sorghum bicolor are interfertile, though outcrossing rates and fertility can be low due to ploidy incompatability \cite{doggett1988sorghum, arriola1996crop}\.

\cite{po1982sorghum}\ states that natural introgression has been an important force in sorghum evolution.
Domesticate-weedy-wild complexes are common when sorghum cultivars are grown in sympatry with wild members of the Eu-Sorghum subgenus \cite{de1978systematics, doggett1968disruptive, baker1972human}\.
Rates of outcrossing within sorghum range from 0-30% \cite{doggett1988sorghum}\.
Natural introgression between wild and domesticated sorghum has been documented in both directions \cite{kuhlman2006interspecific, aldrich1992restriction, aldrich1992patterns, doggett1988sorghum, baker1972migrations}\.
These hybrids express diminished fertility
Members of Sorghum bicolor have adapted to a broad range of altitudinal, precipitation, and temperature clines across Africa and around the world \cite{po1982sorghum, CWR}\.
Breeding efforts have focused on incorporating exotic germplasm (wild sorghums as well as relatives from other genera \cite{de1976cytogenetics}\) for its adaptations to biotic and abiotic stresses \cite{reddy2006current, po1982sorghum, johnson1979breeding}\.
Resistance to greenbug has been introgressed into cultivated sorghum through modern breeding efforts \cite{johnson1979breeding}\.

\end{document}
