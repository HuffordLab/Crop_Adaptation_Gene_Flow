\documentclass[11pt]{article}
\usepackage{color}
\usepackage{authblk}%allows footnote format for authors
\usepackage[letterpaper, margin=1in]{geometry} %package that allows changes in margins and header/footers

\newcommand{\mbh}[1]{\textcolor{green}{ \emph{\scriptsize  #1}} } %creating command for Matt's comments
\newcommand{\jri}[1]{\textcolor{red}{ \emph{\scriptsize  #1}} } %creating command for Matt's comments

\title{Historical demography of maize during diffusion from its center of origin}

\author[1]{Li Wang}%author information
\author[2]{Timothy M. Beissinger}
\author[2,3]{Jeffrey Ross-Ibarra}
\author[1,4]{Matthew B. Hufford}
\affil[1]{Department of Ecology, Evolution, and Organismal Biology, Iowa State University, Ames, Iowa, USA}
\affil[2]{Department of Plant Sciences, University of California Davis, Davis, CA, USA}
\affil[3]{Center for Population Biology and Genome Center, University of California Davis, Davis, CA, USA}
\affil[4]{Corresponding Author: Matthew B. Hufford; 339A Bessey Hall, Iowa State University, Ames, IA, USA; phone: 1-515-294-8511; email: mhufford@iastate.edu}
\date{}

\begin{document}

\maketitle


\section*{Introduction}
\subsection*{Hypothesis on demography of maize diffusion}

Maize as a main crop worldwide, has been mainly studied done on domestication, little on diffusion. Cite van Etten, Vigouroux, van Heerwaarden, Fonseca, Shohei. 
Maize, compared to its wild progenitor, has undergone a domestication bottleneck, during which, its diversity has been largely reduced (Hufford et al. 2012).
Post domestication, maize colonized multiple highland habitats in Americas during its migration. 
When adapted to the highland habitat, does maize population undertake a second bottleneck? 
Or, are they experiencing the serial founder effects with migrating away from its origin center?
Describing the concept of serial founder effects and listed examples of serial founder effects found in Human population history. 

We know little about demography except some introgression (and hints from Ross-Ibarra 2009).
However, the gene exchange between maize and its wild progenitor is not well-known to us. 
Ross-Ibarra et al. (2009) pointed out the evidence of introgression between multiple teosinte species and maize solely based on a dozen of nuclear genes.
Can we identify a signature of introgression when revisiting the question with whole genome sequencing data?
Hufford et al. (2013) discovered the adaptive introgression from \emph{mexicana} to maize in Mexican Highland, but question remained unanswered. 
For example, does this kind of introgression is widespread to other highland populations, even those populations without direct contact with \emph{mexicana} populations ?

\subsection*{}

little about how post-domestication evolution has affected genomic diversity (except in landraces in highlands).


\section*{Methods}

\jri{you can start writing these up already}

\section*{Results}

\subsection*{Diversity and Population Structure}

\subsubsection*{SEEDS data}
\begin{itemize}
\item PCA (Fig1A)
\item diversity moving away from center of origin (Fig 1B)
\end{itemize}

\subsubsection*{whole genome sequence data}

To investigate the genomic impact of diffusion post domesticaiton, we sampled 30 genomes from across the range, including from Andes which are undersampled.  

\begin{itemize}
\item Sampling
\item PCA and ngsadmix (Figure 2A)
\item genetic diversity and tajima's D (Figure2B)
\item ngsF
\jri{not sure this figure 2 is useful, thoughts?}
\end{itemize}

\subsection*{Impact of introgression}
\begin{itemize}
\item D stats with mexicana (Figure 3A is either Dstats with mex or Gmin with mex, but something mex-introgression related)
\item Gmin with mexicana (inv4m, other regions)
\item D stats with parviglumis, luxurians and diploperrennis (compare to R-I 2009, Doebley, etc.) (FIGURE 3B)
\end{itemize}

\subsection*{Demographic change}
\begin{itemize}
\item MSMC -- consistent with SeeDs?, check consistency with sims (Figure 4 of MSMC)
\item dadi?
\item show introgression not messing this up (supplement)
\end{itemize}

\subsection*{Impacts of a recent strong bottleneck in S. Am.}
\begin{itemize}
\item ROH, GERP homozygosity (Figure 5)
\item burden test?
\end{itemize}

\section*{Discussion}

blah blah lots more to learn.




%
%\begin{enumerate}
%\item \textbf{General characterization of populations}
%\begin{itemize}
%\item Population structure
%
%PCA analyses (ngsTools, ngsCovar):  
%the first PC demonstrates the geographic north-south differentiation; Southwest US and Andes populations are at the two extremes.
%\item Genetic diversity ($\hat\theta_\pi$ from ANGSD)
%\begin{itemize}
%\item Andes exhibited significantly lowest genetic diversity
%\item Southwest US, Guatemala High, SA Low, Mexican Low form intermediate, indistinguishable group
%\item Mexican Highland population has the highest diversity
%
%\item Inbreeding coeffecients (ngsF)
%
%No differences among populations observed, F very low for all individuals, which provides little evidence for recent inbreeding.
%
%
%
%Do introgressed regions contribute to the higher diversity in the Mexican Highlands?  
%\jri{make introgression a separate section, include results from luxurians. we have a lane of PE 100 from diploperennis you could use too. iplant/rossibarra/diplo. just ask and i can give you access.  this lets you test some of the arguments made in Ross-Ibarra 2009}
%
%Is there evidence of \emph{mexicana} introgression in other highland populations (\textit{i.e.}, did \textit{mexicana} haplotypes spread to nearby, allopatric highland regions in Guatemala and the Southwest US or are they limited to the Mexican Highlands?)
%
%Detection of introgression (ABBA-BABA and Gmin scan) further validate the introgression between the Mexican Highland population and \textit{Zea mays} ssp. \textit{mexicana}.
%
%\textcolor{blue}
%{To Do: ABBA-BABA statistics and Gmin scan for highland regions outside of Mexico vs. \textit{mexicana}.}
%
%\jri{need to test D against all lowland mex. also include lowland mex. in test.}
%
%
%\textcolor{blue}
%{Plot $\hat\theta_\pi$ in introgressed regions relative to the genome-wide distribution for each population.}
%\end{itemize}
%
%
%
%\item Demography suggested by Tajima's D (ANGSD)
%
%Andes is the only population with median Tajima's D \textgreater{} 0 (significantly more positive than other pops), indicating an absence of rare alleles, and possibly stronger bottleneck.
%\end{itemize}
%
%\item \textbf{What is the pattern of demography in maize populations?}
%\begin{itemize}
%\item MSMC analyses
%
%Our results suggest serial founder effects, but do we trust these given the runaway behavior and smoothing over demographic changes that have previously been observed in PSMC and Dical?
%
%\textcolor{blue}{To Do: Simulation to confirm applicability of method in maize: simulate a continuous population size reduction and two bottlenecks} \textcolor{red}{Tim: also run in dadi to check the likelihoods of serial founder effect model compared to others?}
%
%\item We observe a significant negative correlation between proportion of polymorphic sites and geographic distance from the Balsas using SEEDS data
%\end{itemize}
%
%\item \textbf{The outcomes of an extreme bottleneck in the Andes}
%\begin{itemize}
%
%
%\item Runs of Homozygosity (PLINK)
%
%In general, the Andes shows more runs of homozygosity (ROH), particularly small ROHs ($<1$MB), consistent with a stronger bottleneck in this population. 
%When looking only at larger ROHs ($>1$MB), which might reflect more recent inbreeding events, there is no difference among populations (confirms ngsF result).
%\mbh{Li, we should summarize this result based on genetic distances}
%
%\item Mutation load
%
%The Andes does not show higher mutation load when considering both the heterozygous and homozygous non-reference sites, but does exhibit higher load when considering \emph{ONLY} the non-reference homozygous sites, indicating a potential role of dominance in the mutation load. 
%With a higher GERP cutoff (moderately and strongly deleterious sites), the difference in mutation load between the Andes and other populations is even stronger. 
%Additionally, the Andes has a higher percentage of derived homozygotes (polarized vs. \textit{Tripsacum}), consistent with less efficient selection due to a stronger bottleneck.
%\jri{consider adding the burden ratio test here}
%\end{itemize}
%
%
%\item \textbf{Potential additional analyses}
%\begin{itemize}
%\item \textcolor{blue}{Running MSMC without introgressed regions or with exclusively introgression regions.}
%\end{itemize}
%\end{enumerate}

\end{document}

